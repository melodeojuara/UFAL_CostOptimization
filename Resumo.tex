\documentclass{article}
\usepackage[utf8]{inputenc}

\title{Proposta de cálculo para a estimativa de custos do complexo esportivo da Universidade Federal de Alagoas}
\author{Filipe de Araujo}
\date{Maceió, Al. Janeiro de 2023}

\begin{document}

\maketitle

\section{Resumo}
    \text{A equação abaixo foi criada para realizar a estimativa dos custos do complexo esportivo da UFAL (CE-UFAL).
    As variáveis foram escolhidas de forma a abrangir o máximo de informações disponíveis sobre as edificações.}
    \text{Os custos fixos e variáveis foram abordados três aspectos do CE-UFAL: Depreciação, custos com contratos
    de segurança e limpeza e custos com o consumo de energia elétrica.}
    
    \text{Desta forma, temos:}
\begin{equation}
     Ct = {d*h\left(
     \left(\frac{\sum_{i=1}^{i} (Pi)}{1000}\right) + 
     \left(\frac{(Ai * (\frac{1}{1500}) * Vfl)+Vfs}{Td * Th} \right) + 
     \left(\frac{\frac{ni^2 - xi^2}{n^2} * VCi}{Tm * Td * Th}\right)
     \right)}
\end{equation}

    \text{Onde:
        \\Pi = Potência do circuito i selecionado
		\\d = quantidade de duração do evento em duas
		\\h = quantidade de horas diárias do evento 
		\\Ai = Área em m² do prédio i selecionado
		\\Vfl = Valor para o serviço de limpeza
		\\Vfs = Valor para o serviço de segurança
		\\ni = Vida útil esperada para o prédio i selecionado
		\\xi = Idade do prédio i selecionado
		\\VCi = Valor de construção para o prédio i selecionado
		\\Tm = Total de meses em um ano 
		\\Td = Total de dias em um mês
		\\Th = Total de horas em um dia }

\end{document}
